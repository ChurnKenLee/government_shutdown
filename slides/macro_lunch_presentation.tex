\documentclass[aspectratio=169]{beamer}
%
% Choose how your presentation looks.
%
% For more themes, color themes and font themes, see:
% http://deic.uab.es/~iblanes/beamer_gallery/index_by_theme.html
%
\mode<presentation>
{
    \usetheme{default}      % or try Darmstadt, Madrid, Warsaw, ...
    \usecolortheme{default} % or try albatross, beaver, crane, ...
    \usefonttheme{default}  % or try serif, structurebold, ...
    \setbeamertemplate{navigation symbols}{}
    \setbeamertemplate{caption}[numbered]
}

\usepackage[english]{babel}
\usepackage[utf8]{inputenc}
\usepackage{graphicx}
\usepackage{breqn}
\usepackage{bbm}
\usepackage{multicol}
\usepackage{booktabs}
%\usepackage{enumitem}

\newenvironment{wideitemize}{\itemize\addtolength{\itemsep}{10pt}}{\enditemize}

\DeclareMathOperator*{\argmax}{arg\,max}
\DeclareMathOperator*{\argmin}{arg\,min}

\usepackage[style=authoryear, backend=biber]{biblatex}
\addbibresource{}

\title{Occupation-specific local labor market spillovers of federal government employment}
\author{Nathaniel Bechhofer \and Churn Ken Lee}
\institute{UC San Diego}
\date{}

\begin{document}

\begin{frame}
    \titlepage
\end{frame}

\begin{frame}
    \frametitle{Question}

    Do hiring shocks to the occupational composition of a labor market change the composition by more or less than one-to-one? 

\end{frame}

\begin{frame}
    \frametitle{Short answer}

    \begin{wideitemize}
        \item A 1 percent increase in federal employment in a given occupation-MSA (as a percentage of total employment of that occupation-MSA)
        \item is associated with a 2 to 5 percent decrease in non-federal employment in that same occupation-MSA
    \end{wideitemize}

\end{frame}

\begin{frame}
    \frametitle{Data}

    \begin{wideitemize}
        \item OPM full-count of Non-DoD employees, 1974 - 2014 ($\sim$ 1 million)
        \begin{itemize}
            \item Match to metro area and Census occupation codes
        \end{itemize}
        \item BLS Occupational Employment and Wage Statistics (OEWS) at the metro level
    \end{wideitemize}

\end{frame}

\begin{frame}
    \frametitle{Regression results}

    \begin{table}[width = \textwidth]
        {\centering
        \begin{tabular}{l*{5}{c}}
            \toprule
            &\multicolumn{5}{c}{Percent change in non-federal employment} \\
            &\multicolumn{1}{c}{(1)}&\multicolumn{1}{c}{(2)}&\multicolumn{1}{c}{(3)}&\multicolumn{1}{c}{(4)}&\multicolumn{1}{c}{(5)} \\
            \midrule
            Percent change in federal &      -4.895&      -5.468&      -5.472&      -5.590&      -2.245\\
            emp as percent of total &     (0.411)&     (0.478)&     (0.489)&     (0.515)&     (0.238)\\
            \midrule
            Weight              &          No&         Yes&         Yes&         Yes&         Yes\\
            Year FE             &          No&          No&         Yes&         Yes&         Yes\\
            Year X MSA FE       &          No&          No&          No&         Yes&          No\\
            Year X OCC FE       &          No&          No&          No&          No&         Yes\\
            N                   &       13408&        9309&        9309&        9309&        9309\\
            \bottomrule
            \multicolumn{6}{l}{\footnotesize Standard errors in parentheses}\\
            \end{tabular}
        }
    \end{table}

\end{frame}





\end{document}


